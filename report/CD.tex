\documentclass[11pt,a4paper]{article}
\usepackage[utf8]{inputenc}
\usepackage[hmargin=2.0cm,vmargin=2.5cm,bindingoffset=0.5cm]{geometry}
\usepackage{amsfonts}
\usepackage{amsmath,amsthm,amssymb}
\usepackage{hyperref}
\usepackage{graphicx}
\usepackage{tikz}
\usepackage{mathtools}
\DeclarePairedDelimiter\ceil{\lceil}{\rceil}
\DeclarePairedDelimiter\floor{\lfloor}{\rfloor}
%\usepackage{float}
\usepackage{placeins}
\usepackage{diagbox}
\newtheorem{thm}{Theorem}
\usepackage{subcaption}
%\usepackage{subfigure}
\usepackage[english]{babel}
\author{Mohit}
\title{Counterdiabatic driving}
\begin{document}
\maketitle

\section{Introduction}

Our Hamiltonian would be controlled using a control parameter called $\lambda$. Our aim would be drive the system without any transition.
\section{Adiabatic gauge potential: derivation}
Let Hamiltonian $H_0(\lambda (t))$ satisfy the following equation

\begin{equation}
H_0(\lambda (t)) |\psi \rangle= i \partial_t|\psi \rangle
\end{equation}

Let us go to rotating frame so as to diagonalize our Hamiltonian. Required unitary transformation $U(\lambda)$ would depend on parameter $\lambda$. Wave function in moving frame is $|\tilde{\psi}  \rangle = U^{\dagger} |\tilde{\psi}  \rangle$. In this basis, Hamiltonian is diagonal:
$\tilde{H_0}= U^{\dagger} H_0 U = \sum_n \epsilon (\lambda)  |n (\lambda)\rangle \langle  n (\lambda) |$. \footnote{Note that expectation value should remain same in both basis, i.e.$ \langle \tilde{\psi} | \tilde{H_0}  |\tilde{\psi}  \rangle= \langle{\psi} | {H_0}  |{\psi}  \rangle$}

How does the wave function evolve in new basis?
\begin{equation}
 i \partial_t|\tilde{\psi} \rangle=(\tilde{H_0
 }(\lambda (t)) - \dot{\lambda} \tilde{\mathcal{A_\lambda}}) |\psi \rangle
\end{equation}
\section{Free interacting fermions in an external potential}

\begin{equation}
H_0= -J \sum_{j=1}^{L-1} (c^{\dagger}_j c_{j+1} +c^{\dagger}_{j+1} c_{j}) + \sum_{j=1}^{L} V_j(\lambda) c^{\dagger}_jc_j
\end{equation}


\begin{equation}
\mathcal{A}^*_{\lambda}= i  \sum_{j=1}^{L-1} \alpha_j (c^{\dagger}_j c_{j+1} - c^{\dagger}_{j+1} c_{j}) 
\end{equation}
\appendix

\section{Spin 1/2 particle in a time-dependent magnetic field}
I would include a derivation from lecture notes to gain an intuition her. I also plan to understand Berry's paper and reproduce some of his calculations in this appendix.
\end{document}