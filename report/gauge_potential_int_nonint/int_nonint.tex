\documentclass[11pt,a4paper]{article}
\usepackage[utf8]{inputenc}
\usepackage[hmargin=2.0cm,vmargin=2.5cm,bindingoffset=0.5cm]{geometry}
\usepackage{amsfonts}
\usepackage{amsmath,amsthm,amssymb}
\allowdisplaybreaks
\usepackage{hyperref}
\usepackage{graphicx}
\usepackage{tikz}
\usepackage{mathtools}
\DeclarePairedDelimiter\ceil{\lceil}{\rceil}
\DeclarePairedDelimiter\floor{\lfloor}{\rfloor}
%\usepackage{float}
\usepackage{placeins}
\usepackage{diagbox}
\DeclareMathOperator{\Tr}{Tr}
\newtheorem{thm}{Theorem}
\usepackage{subcaption}
%\usepackage{subfigure}
\usepackage[english]{babel}
\author{Mohit}
\title{Adiabatic gauge potential of quantum integrable and non-integrable systems  }
\begin{document}
\maketitle
%\tableofcontents

\section{Introduction}


Adiabatic gauge potentials are useful for controlling a quantum system when it's driven externally from one configuration to another. These potentials help us in  circumventing standard adiabatic limitations which requires infinitesimally small rates \cite{demirplak2003adiabatic,demirplak2005assisted, berry2009transitionless}. For example, these potentials can be used for arbitrarily fast annealing protocols and implementing fast dissipationless driving. 



The scaling of norm of gauge potential with system's size is quite different for  quantum integrable and non-integrable systems. On one hand, for integrable systems , exact gauge potential are supposed to scale like a polynomial in system size. This is due to extensive number of symmetries that exist and as a result, they have a ``lot" of degenerate energy levels which comes with their respective ``selection rules".  This can be easily seen  for Transverse Ising model whose analytical expression of gauge potential is known in literature.

On the other hand, for non-integrable systems, using  Eigenstate Thermalization Hypothesis (ETH)\cite{d2016quantum} , we can show that norm of exact gauge potential scale exponentially in system size. This can be verified numerically using exact diagonalization on spin system upto size $L=15$. 


We can exploit this property to distinguish  between quantum integrable and non-integrable system. Our method should be better than conventional method (energy level distribution) used in literature for this purpose because unlike the conventional method, we don't have to worry about removing symmetry.


%The goal, as of now, is to distinguish between integrable and non-integrable many-body quantum system by studying their approximate gauge adiabatic potential\footnote{We expect results to be valid for classical system too. But for now, we would focus on quantum systems.}

 



\section{Adiabatic gauge potential}
Adiabatic gauge potentials are the generators of a unitary transformation which diagonalize the instantaneous Hamiltonian, attempting to leave its eigenbasis invariant as the parameter is changed. These adiabatic gauge potentials generate \textit{non-adiabatic} corrections to Hamiltonian in the moving basis ($\lambda$ -dependent basis).
 
 This is something from Anatoli's lecture notes \cite{kolodrubetz2016geometry}--
``an adiabatic basis is a family of adiabatically connected eigenstates, i.e., eigenstates related
to a particular initial basis by adiabatic (infinitesimally slow) evolution of the parameter $\lambda$. For example, if two levels cross they will exchange order energetically but the adiabatic connection will be non-singular."


$H (\lambda) |n(\lambda) \rangle = E_n (\lambda) |n(\lambda) $. Let's derive diagonal and off-diagonal elements. 


\begin{itemize}
\item \textbf{n-th diagonal element:} $A_{\lambda}^n= \langle n |A_{\lambda} | n \rangle=  i \hbar\langle n |\partial_{\lambda} | n \rangle $
\item \textbf{off- diagonal element:} We use the identity $\langle m |H(\lambda) | n \rangle=0 \quad, n \neq m$ and then differentiate with respect to $\lambda$ to obtain:
\begin{align}
\boxed{\langle m |A_{\lambda} | n \rangle =  -i \hbar \dfrac{\langle m |\partial_{\lambda}H | n \rangle}{E_m-E_n}}
\end{align}
where both  energies ($E_m, E_n$) and eigenvectors ($|m \rangle, |n \rangle$) depend on $\lambda$.
\end{itemize}







\subsection{Eigenstate Thermalization Hypothesis}
Eigenstate Thermalization Hypothesis( ETH) gives us an ansatz for matrix elements of observables in the basis of energy eigenstates  \cite{d2016quantum}:
\begin{equation}
O_{mn}= O( \bar{E}) \delta_{mn} + e^{-S(\bar{E})/2} f_O(\bar{E}, \omega) R_{mn}
\end{equation}
where $\bar{E}= (E_m +E_n)/2, \omega= E_n- E_m$ and $S(E)$ is the thermodynamic entropy at energy $E$.

We note that it's applicable only for few-body operators of a non-integrable Hamiltonian. By few-body, we mean $n$ body observables with $n \ll N$, where $N$ is the total number of spins, particles, etc. For example, projection operator to eigenstates of many body  Hamiltonian $\hat{P}_{\alpha}= |\Psi_{\alpha} \rangle \langle\Psi_{\alpha}  |$ don't satisfy ETH and it also doesn't satisfy predictions of statistical mechanics. Why is that? We expect that microcanonical averaging should be equivalent to canonical averaging:
\begin{equation}
 \langle\Psi_{\alpha}  |O|\Psi_{\alpha} \rangle= \dfrac{\Tr O e^{-\beta H}}{\Tr e^{-\beta H}}
\end{equation}
We can see $O= \hat{P}_{\alpha}$ doesn't satisfy the above equation (since left hand side is one and the trace of right hand side can be computed in energy basis to find that it's not one). Projection operator is non-local in real space, and we argue that this is the reason it doesn't satisfy ETH and is not experimentally measurable.




\section{Norm of adiabatic gauge potential}
Let's compute the norm by noting that $A_{\lambda}$ has only off-diagonal elements in energy basis in our gauge choice:

\begin{align}
||A_{\lambda}||^2 &= \Tr  A_{\lambda}^2 \\
&= \sum_n \langle n | A^2_{\lambda}| n \rangle \\
&= \sum_{n} \langle n | A_{\lambda}| n \rangle ^2 + \sum_n \sum_{m \neq n}  |\langle m | A_{\lambda}| n \rangle|^2 \\
&=  \sum_n \sum_{m \neq n}  |\langle m | A_{\lambda}| n \rangle|^2 \\
&= \hbar^2 \sum_n \sum_{m \neq n}  \dfrac{|\langle m | \partial_{\lambda}H| n \rangle|^2}{(E_m-E_n)^2} 
\end{align}

Hence, in general, for both integrable and non-integrable systems we have: 
\begin{equation}
\boxed{||A_{\lambda}||^2 = \hbar^2\sum_n \sum_{m \neq n}  \dfrac{|\langle m | \partial_{\lambda}H| n \rangle|^2}{(E_m-E_n)^2} }
\end{equation}


\section{Integrable model}
Our goal is to study a integrable model, which is called \textbf{Transverse Field Ising model}. It shows quantum phase transition between ferromagnetic and paramagnetic phases. Moreover, it satisfies Ising symmetry $G= \Pi_i \sigma_i^z$ since $[H, G]=0$, where $H$ is the Hamiltonian.
This model can be written in terms of non-interacting spinless fermions ($c_i, c^{\dagger}_i $) using Jordan- Wigner transformation. 

It's Hamiltonian in spin basis is given by:
\begin{equation}
H= -J \sum_{j} \sigma_j^x \sigma_{j+1}^x - \lambda \sum_{j} \sigma_j^z 
\label{xx_z}
\end{equation}
where we have not specified boundary conditions and $\lambda$ is externally-controlled transverse magnetic field.

This model can be written in terms of non-interacting spinless fermions ($c_i, c^{\dagger}_i $) using Jordan- Wigner transformation: $\sigma_i^z \sim 1 - 2 c^{\dagger}_i c_i   $ and $\sigma_i^+ \sim \prod_{j<i} \sigma_j^z c_j   $. Details can be found elsewhere \cite{sachdev2007quantum} \footnote{Momentum operator chosen to get real valued Hamiltonian is $c_k= \frac{e^{i \pi/4}}{\sqrt{L}}\sum_j c_j e^{-ikj}$, where k is $n\pi/L$ with $n=0,1,2, \ldots L-1$} . Here is what we get after this transformation:

\begin{equation}
\mathcal{H}= \sum_k \psi_k^{\dagger} H_k \psi_k , \quad H_k=-\begin{bmatrix}
     \lambda - \cos k & \sin k \\
\sin k & -(\lambda - \cos k)\\
\end{bmatrix}
\end{equation}
where $\psi_k^{\dagger}= (c^{\dagger}_k, c_{-k})$ is Nambu spinor basis. We can write $H_k$ in terms of Pauli sigma matrices:
\begin{equation}
H_k= -(\lambda - \cos k) \sigma^z_k - \sin k  \sigma^x_k
\end{equation}



Now using our regulator method (whose details are not given in this report), we can obtain :
\begin{equation}
\boxed{A_{\lambda}= \sum_{l=1}^{L} \alpha_l O_l \quad \mbox{where} \quad \alpha_l= -\dfrac{1}{4 L} \sum_k \dfrac{\sin(k) \sin(lk)}{(\cos k - \lambda)^2 + \sin^2 k}}
\end{equation}
where  $O_l$ is given by
\begin{equation}
O_l= 2 i \sum_j (c^{\dagger}_{j} c^{\dagger}_{j+l} - \mbox{h.c})= \sum_j ( \sigma_j^x \sigma_{j+1}^z \ldots \sigma_{j+l-1}^z \sigma_{j+l}^y +  \sigma_j^y \sigma_{j+1}^z \ldots \sigma_{j+l-1}^z \sigma_{j+l}^x)
\end{equation}
This matches with the result already known in literature \cite{del2012assisted, kolodrubetz2016geometry}.


Let's write a first few terms of $O_l$ here:
\begin{align*}
O_{l=1}&=  \sum_j ( \sigma_j^x  \sigma_{j+1}^y +  \sigma_j^y  \sigma_{j+1}^x) \\
O_{l=2} &=  \sum_j ( \sigma_j^x \sigma_{j+1}^z \sigma_{j+2}^y +  \sigma_j^y \sigma_{j+1}^z \sigma_{j+2}^x) \\
\end{align*}


For large enough system size $L$, we can compute $\alpha_l$ \cite{kolodrubetz2016geometry} by computing the sum into an integral and obtain the value of $\alpha_l$ as: 
\begin{equation}
 \alpha_l=\left\{
  \begin{array}{@{}ll@{}}
    h^{l-1} & (h^2 <1), \\
     h^{-l-1} & (h^2 >1)
  \end{array}\right.
\end{equation}



Let's compute norm of gauge potential:
\begin{align}
||A_{\lambda}||^2 &= \Tr  A_{\lambda}^2 \\
&= \Tr \sum_{l,p}  \alpha_p \alpha_l O_l  O_p\\
&=  \sum_{l,p}  \alpha_p \alpha_l  \Tr O_l  O_p\\
&=  \sum_{l=1}^{L}  \alpha_l ^2
\end{align}


Now since $\alpha_l$ for large enough $L$ is exponentially suppressed in $l$, we can argue that


\begin{equation}
\boxed{||A_{\lambda}||^2 \sim  L }
\end{equation}

\section{Non-integrable model}

If we introduce longitudinal magnetic field in Transverse Ising model, then integrability is broken and we get a non-integrable model. We plan to study both local  and global integrability-breaking term. 

\begin{equation}
H= -J \sum_{j} \sigma_j^x (\sigma_{j+1}^x+ \sigma_{j-1}^x) - h\sum_{j} \sigma_j^z -\lambda \sum_{j} \sigma_j^x 
\end{equation}
In this model, $\partial_{\lambda}H = - \sum_{j} \sigma_j^x  $ is a global operator.


\begin{equation}
H= -J \sum_{j} \sigma_j^x (\sigma_{j+1}^x+ \sigma_{j-1}^x) - h\sum_{j} \sigma_j^z -\lambda  \sigma_0^x 
\end{equation}
In this model, $\partial_{\lambda}H = -  \sigma_0^x  $ is a local operator.


\subsection{ETH applied to norm}
$\partial_{\lambda}H$ may or may not be a local operator. We would be studying such non-integrable models in which it is a local operator. Hence, we can apply ETH on the operator $\partial_{\lambda}H$.

\subsubsection{Heuristic argument}
\begin{align*}
||A_{\lambda}||^2= \hbar^2\sum_n \sum_{m \neq n}  \dfrac{|\langle m | \partial_{\lambda}H| n \rangle|^2}{\omega_{mn}^2}
\end{align*}
where $\omega_{mn}= E_m-E_n$.
We would argue that the biggest contribution to norm would come from the smallest $\omega_{mn}$ because it's exponentially small in system size. Hence, we find that using ETH for  $\partial_{\lambda}H$:

\begin{align*}
||A_{\lambda}||^2&= \hbar^2\sum_n \sum_{m \neq n}  \dfrac{|\langle m | \partial_{\lambda}H| n \rangle|^2}{\omega_{mn}^2}\\
&= \hbar^2\sum_n \sum_{m \neq n}  \dfrac{ e^{-S}}{e^{-2S}} \\
&= \hbar^2\sum_n \sum_{m \neq n}  e^{S} \\
& \simeq \hbar^2 2^L  e^{L} 
\end{align*}
where we have used the fact that entropy is extensive, i.e. $S \sim L$. Hence, norm averaged over system size is exponential in system size with $\hbar=1$
\begin{equation}
\boxed{||A_{\lambda}||^2/ 2^N \sim  e^{L} 
}
\end{equation}

Exponential scaling with system size of gauge potential is due to exponential small eigenvalues. Since these eigenvalues appear in the denominator of gauge potential expression, it's called \textbf{zero denominator problem} in literature \cite{kolodrubetz2016geometry}. 

In Dries's notes, you would find how we are attempting to solve this problem.

\subsubsection{Formal calculation}
For formal calculation, I would need to introduce a cutoff $\mu$. Otherwise, norm diverges in thermodynamic limit $L \rightarrow \infty$, which is clear from above heuristic arguments.

\begin{eqnarray}
\langle n | A_{\lambda} | m \rangle &=& \lim_{\mu \rightarrow 0} \lim_{L \rightarrow \infty } -i \hbar \dfrac{\langle n | \partial_{\lambda}H  | m \rangle}{(E_n-E_m)^2 + \mu^2} (E_n-E_m) 
\label{off-digonal}
\end{eqnarray}
where we have chosen a gauge choice in which diagonal elements are zero in energy basis, i.e. $A_{\lambda}^{nn}=0$. 

\begin{equation}
||A_{\lambda}||^2 = \hbar^2\sum_n ||A_{\lambda}||^2_{nn}
\end{equation}
where $||A_{\lambda}||^2_{nn} =\sum_{m \neq n}  \dfrac{(E_m-E_n)^2}{((E_m-E_n)^2 + \mu^2)^2} |\langle m | \partial_{\lambda}H| n \rangle|^2$.

Let's simplify this using ETH:
\begin{align*}
||A_{\lambda}||^2_{nn} &= \sum_{m \neq n}  \dfrac{(E_m-E_n)^2}{((E_m-E_n)^2 + \mu^2)^2} |\langle m | \partial_{\lambda}H| n \rangle|^2\\
&=\sum_{m \neq n}  \dfrac{\omega^2}{(\omega^2 + \mu^2)^2} e^{-S(\bar{E})} |f_O(\bar{E}, \omega) R_{mn}|^2\\
&=\sum_{m \neq n}  \dfrac{\omega^2}{(\omega^2 + \mu^2)^2} e^{-S(E_n -\omega/2)} |f_O(E_n - \omega/2, \omega)|^2 |R_{mn}|^2
\end{align*}
where $\bar{E}= (E_m +E_n)/2=E_n - \omega/2$ ,$ \omega= E_n- E_m$ and $S(E)$ is the thermodynamic entropy at energy $E$.
We would need to convert the sum into integral where we use the fact that function $f_O$ is smooth and fluctuations of $|R_{mn}|^2$ average out in the sum.

\begin{equation}
\sum_{m \neq n}  \rightarrow \int d \omega \Omega(E_{n}- \omega)= \int d \omega e^{S(E_{n}- \omega)}
\end{equation}
where $\Omega(E_{n}+ \omega)$ is density of states.


\begin{align*}
||A_{\lambda}||^2_{nn} &=\int d \omega e^{S(E_{n}- \omega)-S(E_n -\omega/2)} \dfrac{\omega^2}{(\omega^2 + \mu^2)^2}  |f_O(E_n - \omega/2, \omega)|^2 
\end{align*}
$S(E_{n}- \omega)-S(E_n -\omega/2) = -\beta \omega/2 + \ldots $ and $f_O(E_n - \omega/2, \omega)=f_O(E_n, \omega) + \ldots $ we have

\begin{align*}
||A_{\lambda}||^2_{nn} &=\int_a^{b} d \omega e^{-\beta \omega/2} \dfrac{\omega^2}{(\omega^2 + \mu^2)^2}  |f_O(E_n, \omega)|^2 
\end{align*}
where $a$ represents the minimum energy difference $E_m-E_n$ in thermodynamic limit (which is energy gap between $n$-th state and nearest eigenstate (either $E_{n+1}$ or $E_{n-1}$ depending upon what is $\min \{|E_{n-1}-E_{n}|,|E_{n+1}-E_{n}| \}$) and $b$ is the maximum energy difference (for which we have to find $m$-th state such that we get $\max \{|E_{m}-E_{n}|\}$) ). $a= e^{-S} \sim e^{-\delta L}$ and $b= \gamma L$, where $\gamma$ and $\delta$ are constants that depend on the details of Hamiltonian.

Let's denote $I=e^{-\beta \omega/2} \dfrac{\omega^2}{(\omega^2 + \mu^2)^2}$ and find out how it depends on L. First, we check on upper limit.

\begin{align*}
\lim_{L \rightarrow \infty}I(\omega=L)=\lim_{L \rightarrow \infty} e^{-\beta L/2} \dfrac{L^2}{(L^2 + \mu^2)^2} \rightarrow 0
\end{align*}
Now on lower limit.
\begin{align*}
\lim_{L \rightarrow \infty}I(\omega=e^{-L})=\lim_{L \rightarrow \infty} e^{-\beta e^{-L}/2} \dfrac{e^{-2L}}{(e^{-2L} + \mu^2)^2}=\lim_{L \rightarrow \infty} \dfrac{e^{-2L}}{(e^{-2L} + \mu^2)^2}
\end{align*}

\begin{equation}
  \lim_{L \rightarrow \infty}I(\omega=e^{-L})=\left\{
  \begin{array}{@{}ll@{}}
    e^{2L} & (\mu^2 \ll e^{-2L}), \\
     \dfrac{e^{-2L}}{ \mu^4} & (\mu^2 \gg e^{-2L})
  \end{array}\right.
\end{equation}

Now, let's compute the norm while assuming $ |f_O(E_n, \omega)|^2 $ is a constant in $\omega$. Hence, we get:
\begin{align*}
||A_{\lambda}||^2_{nn} &=  |f_O(E_n)|^2 \int_0^{\infty} d \omega e^{-\beta \omega/2} \dfrac{\omega^2}{(\omega^2 + \mu^2)^2} 
\end{align*}
Let's assume $\beta \ll 1$ (high temperature limit):

\begin{align*}
||A_{\lambda}||^2_{nn} &=  |f_O(E_n)|^2 \int_0^{\infty} d \omega \left(1-\beta \omega/2 + \ldots \right) \dfrac{\omega^2}{(\omega^2 + \mu^2)^2}  \\
&=|f_O(E_n)|^2\left(\dfrac{\pi}{4 \mu}-\dfrac{\beta }{4}-\dfrac{\beta }{4} \log (\mu^2+ \omega^2)|_0^{\infty}  + \ldots \right) \\
\end{align*}

Hence, we see that there is a logarithmic divergence for high temperature limit. We also note that there are two limits, in which we find that there is no ultraviolet divergence: $\beta =0$ limit gives $\pi/ 4 \mu$ and $\beta \rightarrow \infty$ limit gives us zero norm. I don't understand why zero temperature limit gives zero norm.


\section{Norm computed using ED}

Let's look at the expression of off-diagonal elements of gauge potential:
\begin{equation}
\langle m |A_{\lambda} | n \rangle =  -i \hbar \dfrac{\langle m |\partial_{\lambda}H | n \rangle}{E_m-E_n}, \quad n \neq m
\end{equation}
We see that while using ED, we need to be wary of degenerate eigenvalues. Do these degenerate eigenvalues contribute to norm of gauge potential?

Let's consider $H | n(\lambda) \rangle= E_n  | n(\lambda) \rangle $. Hence, we have $\langle m(\lambda)  |  H | n(\lambda) \rangle=0$ for $n\neq m$. We can exploit this property to get some insight:
\begin{align*}
\partial_{\lambda}\langle m  |  H | n\rangle&=0\\
\langle \partial_{\lambda} m |  H | n \rangle + \langle  m  |  H |\partial_{\lambda} n \rangle + \langle  m  | \partial_{\lambda} H | n \rangle&=0\\
\langle \partial_{\lambda} m   | n \rangle E_n + E_m\langle  m  |   \partial_{\lambda} n \rangle + \langle  m  | \partial_{\lambda} H | n \rangle&=0\\
(E_n - E_m)\langle  \partial_{\lambda} m  |    n \rangle + \langle  m  | \partial_{\lambda} H | n \rangle&=0
\end{align*}
Hence, we find that if there are two degenerate energy levels $n$ and $m$ such that $E_n=E_m$, then $\langle  m  | \partial_{\lambda} H | n \rangle=0$. Hence, the contribution to norm of gauge potential from this pair of energy levels will be zero. I should check this numerically if results of my code respect this property.


We have two options while computing norm using ED: either take out all degenerate eigenvalues and work in non-degenerate eigen-subspace or introduce a cutoff $\mu$ to help us out here. We find that second option is easier to work numerically. Our $\mu$-dependent gauge potential is given by:
\begin{equation}
\langle m |A_{\lambda} | n \rangle =  -i \hbar \dfrac{\langle m |\partial_{\lambda}H | n \rangle}{\omega_{mn}^2+ \mu^2} \omega_{mn}
\end{equation}
We note that for $\mu \neq 0$, $A_{\lambda}$ is zero for degenerate levels $n$ and $m$. What should be the value of $\mu$ chosen?  If $\mu^2 \ll \omega_{mn}^2$, then we would be back to square one because then we can ignore $\mu^2$ in denominator, and then the computer would complain of division by zero problem for degenerate levels $n$ and $m$. If $\mu^2 \gg \omega_{mn}^2$, then we would get approximate gauge potential, not the exact one. 

Numerically, we choose $\mu^2$ to be the difference between two closest  floating point numbers.

\appendix




\bibliography{ref} 

\bibliographystyle{unsrt}


\end{document}
