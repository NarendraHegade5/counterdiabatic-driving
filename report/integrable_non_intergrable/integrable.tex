\documentclass[11pt,a4paper]{article}
\usepackage[utf8]{inputenc}
\usepackage[hmargin=2.0cm,vmargin=2.5cm,bindingoffset=0.5cm]{geometry}
\usepackage{amsfonts}
\usepackage{amsmath,amsthm,amssymb}
\usepackage{hyperref}
\usepackage{graphicx}
\usepackage{tikz}
\usepackage{mathtools}
\DeclarePairedDelimiter\ceil{\lceil}{\rceil}
\DeclarePairedDelimiter\floor{\lfloor}{\rfloor}
%\usepackage{float}
\usepackage{placeins}
\usepackage{diagbox}
\newtheorem{thm}{Theorem}
\usepackage{subcaption}
%\usepackage{subfigure}
\usepackage[english]{babel}
\author{Mohit}
\title{Integrable and non-integrable systems}

\begin{document}
\maketitle


The goal, as of now, is to distinguish between integrable and non-integrable many-body quantum system by studying their approximate gauge adiabatic potential\footnote{We expect results to be valid for classical system too. But for now, we would focus on quantum systems.}

 
Classically, on one hand, integrable systems have a lot of constants of motion, and as a result, they have a few independent degrees of freedom. On the other hand, non-integrable systems contain a large number of independent degrees of freedom. We expect a similar picture for quantum systems.

The central idea is to apply Eigenstate Thermalization Hypothesis (ETH) to operators of exact gauge potential in non-integrable quantum systems, and claim that its' norm scales exponentially in system 
size. Whereas for integrable systems, exact gauge potential are supposed to scale like a polynomial in system size.

What is an integrable quantum systems? To the best of my knowledge, the general definition of integrability for quantum systems has not been reached conclusively. Despite this, there are some models which are commonly agreed to be integrable and similarly, there are model which are called non-integrable in literature. For our purposes, we would use such models to get some intuition.

Let's list down a few properties of \textbf{integrable} quantum systems: 
\begin{itemize}
\item The many-body density matrix of those systems don't thermalize to Gibbs distribution. In fact, they thermalize to a generalized Gibbs distribution. (see \cite{rigol2007relaxation,cassidy2011generalized} for detail) 
\item They can be diagonalized using a transformation that is local in space\footnote{According to Dries, for 2D transverse quantum Ising model, Jordan Wigner transformation exists to diagonalize the Hamiltonian. However, it's still called a non-integrable model since then the transformation becomes non-local. I need to dig relevant paper for details}. Examples are non-interacting fermions, 1 D Ising model and 1D transverse field Ising model (TFIM).  These can be diagonalized using Bogoliubov, transfer matrix method and Jordan-Wigner transformation, respectively.
\item ETH doesn't apply to them 
\item Distribution of Energy level spacing follows Poisson distribution --energy level attraction.
\end{itemize}

We note here that many body localized (MBL) system is a new kind of integrable system. To understand its' similarity and difference from integrable system, I am quoting a paragraph from \cite{o2016explicit} : 

``In order to explain the basic phenomenology of MBL systems, including their failure to thermalise, a picture of Local Integrals of Motion (LIOMs) has been put forward. According to this picture, the basic mechanism of MBL is similar to integrable models: there emerges an extensive number of operators (“conserved charges”) $\tau_i$ , which commute amongst  themselves $[\tau_i , \tau_j ]=0$ as well as with the Hamiltonian $[H, \tau_i]=0$.

A special property of MBL systems is that $\tau_i$ have eigenvalues $\pm 1$, thus they resemble the bare spin-1/2 operators, and generically there are L such operators in a lattice system of size L. This means that any Hamiltonian eigenvector can be specified by the conserved quantum numbers corresponding to operators $\tau_i$. Because of this extensive number of emergent quantum numbers (that by definition do not change during unitary evolution), the thermalisation of the system is prevented as the MBL state retains the memory of its initial condition. \textbf{The difference between integrable models and MBL systems is in the form of individual $\tau_i$ : in the integrable case, each $\tau_i$. is an extended sum of local operators, while in the MBL case each $\tau_i$. is a single local operator, up to corrections that vanish exponentially with distance to the core.} The subleading (exponentially suppressed) corrections are important, as they cause the distinction between Anderson and MBL insulators. For example, the presence of tails in LIOMs is responsible for the dephasing dynamics and the spreading of entanglement in MBL systems, which does not occur in Anderson insulators"

In \cite{abanin2017recent}, following Hamiltonian is considered for studying MBL:
\begin{equation}
\hat H = -\frac{J}{2} \sum_{j=1}^{N-1} ( \sigma_j^x \sigma_{j+1}^x +  \sigma_j^y \sigma_{j+1}^y) +  V \sum_{j=1}^{N-1} \sigma_j^z \sigma_{j+1}^z +  \sum_{j=1}^{N} h_j\sigma_j^{z}) 
\end{equation}
where $h_j$ is random magnetic field chosen from uniform distribution, i.e. $h_j \in [-W,W]$. In this model, form of $\tau_i$ is given as 
\begin{equation}
\tau_i^z= \sigma_i^z + \sum_{j,k} \sum_{a,b=x,y,z} f^{a,b}_{i,j,k} \sigma_j^a \sigma_k^b
\end{equation}
where weights decay exponentially with distance:
\begin{equation}
f^{a,b}_{i,j,k} \propto \exp(- \max\{ |i-j|, |i-k| \}/ \xi)
\end{equation}
Let's list down a few properties of \textbf{non-integrable} quantum systems: 
\begin{itemize}

\item They cannot be diagonalized using a transformation that is local in space. This is not a strong argument because it just means that such a transformation has not been found yet.
\item ETH does apply to them (\cite{d2016quantum}, \cite{rigol2008thermalization})
\item Distribution of Energy level spacing are correlated and therefore, they show level repulsion. They follow Wigner-Dyson or similar distributions, depending upon the details of Hamiltonian. These properties can be derived using Random Matrix Theory.
\end{itemize}


We do note that both integrable and non-integrable show quantum phase transition \footnote{Is there any difference between phase transitions shown between integrable and non-integrable models? Apparently no.}. An example of quantum phase transition in integrable model: TFIM show paramagnetic-ferromagnetic quantum phase transition.



\bibliography{ref} 

\bibliographystyle{plain}


\end{document}