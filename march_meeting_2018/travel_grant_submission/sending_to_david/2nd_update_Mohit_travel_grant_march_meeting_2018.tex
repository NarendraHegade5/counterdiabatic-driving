\documentclass[11pt,a4paper]{article}
\usepackage[utf8]{inputenc}
\usepackage[hmargin=2.0cm,vmargin=2.5cm,bindingoffset=0.5cm]{geometry}
\usepackage{amsfonts}
\usepackage{amsmath,amsthm,amssymb}
\usepackage[english]{babel}
\author{Mohit Pandey}
\title{Approximate counter-diabatic driving protocols for non-integrable quantum systems  }
\begin{document}
\maketitle
\textbf{Student's statement  providing the context of the contribution}


Quantum computers can provide substantial advantage over classical computers in certain tasks like large integer factorization. The "quantum supremacy" is largely due to the quantum algorithms which use bizarre world of quantum physics to their advantage. Many of these algorithms are dependent on adiabatic processes. For example, quantum adiabatic dynamics can be used for solving satisfiability problem [1], quantum search algorithms[2] and  holonomic quantum computation [3]. 

However, quantum adiabatic theorem imposes restriction due to which systems have to be driven sufficiently slow making it difficult to protect their feeble coherence during long adiabatic dynamics. As a result, we need alternative driving protocols which are robust to noise and decoherence from environment. 

Counter-diabatic (CD) driving protocols, which are also known as "shortcuts-to-adiabaticity" provide the needed robustness. Within this field of CD driving protocol, there is a problem of zero denominator  [4] which prevents us from constructing CD Hamiltonian for non-integrable quantum systems (which are also called quantum chaotic systems). The reason behind this is their exponential sensitivity to perturbations of the Hamiltonian. This makes the quantum control and quantum state transfer of such systems extremely difficult.

Our approximate CD protocols for these systems avoids this problem. Further, our method has a potential of being used as a diagnostic tool for differentiating between quantum integrable and non-integrable systems. Compared to the conventional method, which uses statistics of nearest neighbor energy spacing distribution, our method should work without worrying about identifying the symmetries of the Hamiltonian. 

[1] https://arxiv.org/abs/quant-ph/0001106.

[2]https://arxiv.org/abs/quant-ph/0107015.

[3]Phys. Lett. A 264, 94–99 (1999).

[4]Physics Reports 697 (2017): 1-87.


\end{document}
