\documentclass[11pt,a4paper]{article}
\usepackage[utf8]{inputenc}
\usepackage[hmargin=2.0cm,vmargin=2.5cm,bindingoffset=0.5cm]{geometry}
\usepackage{amsfonts}
\usepackage{amsmath,amsthm,amssymb}
\usepackage[english]{babel}
\author{Mohit Pandey}
\title{Approximate counter-diabatic driving protocols for non-integrable quantum systems  }
\begin{document}
\maketitle
\textbf{Student's statement  providing the context of the contribution}

My present research is going to be useful for the field of quantum computation, quantum simulation of many-body phenomenon, quantum state control and quantum chaos. 

Counter-diabatic (CD) driving protocols, which are also known as "shortcuts-to-adiabaticity", can help us in doing quantum state transfer and building scalable quantum computers, which are robust to noise and decoherence from environment. Within this field of CD driving protocol, there is a problem of zero denominator  [1] which prevents us from constructing CD Hamiltonian for non-integrable quantum systems (which are quantum analogue of classical chaotic systems). The reason behind this is their exponential sensitivity to perturbations of the Hamiltonian. This makes the quantum control of such systems extremely difficult.

Our approximate CD protocols for these systems avoids this problem, which provides an alternative method to variational approximation scheme recently introduced in [2]. Further, our method has a potential of being used as a diagnostic tool for differentiating between quantum integrable and non-integrable systems. Compared to the conventional method, which uses statistics of nearest neighbor energy spacing distribution, our method should work without worrying about identifying the symmetries of the Hamiltonian. Moreover, this method also offers us the potential of learning more about eigenstate thermalization hypothesis, which describe quantum chaotic systems. 



[1] Kolodrubetz, Michael, et al. "Geometry and non-adiabatic response in quantum and classical systems." Physics Reports 697 (2017): 1-87.

[2] Sels, Dries, and Anatoli Polkovnikov. "Minimizing irreversible losses in quantum systems by local counterdiabatic driving." Proceedings of the National Academy of Sciences (2017): 201619826.



\end{document}
