\documentclass[11pt,a4paper]{article}
\usepackage[utf8]{inputenc}
\usepackage[hmargin=2.0cm,vmargin=2.5cm,bindingoffset=0.5cm]{geometry}
\usepackage{amsfonts}
\usepackage{amsmath,amsthm,amssymb}
\allowdisplaybreaks
\usepackage{hyperref}
\usepackage{graphicx}
\usepackage{tikz}
\usepackage{mathtools}
\DeclarePairedDelimiter\ceil{\lceil}{\rceil}
\DeclarePairedDelimiter\floor{\lfloor}{\rfloor}
%\usepackage{float}
\usepackage{placeins}
\usepackage{diagbox}
\DeclareMathOperator{\Tr}{Tr}
\newtheorem{thm}{Theorem}
\usepackage{subcaption}
%\usepackage{subfigure}
\usepackage[english]{babel}
\author{Mohit Pandey, Dries Sels and D. K. Campbell}
\title{Approximate counterdiabatic driving protocols for quantum non-integrable systems  }
\begin{document}
\maketitle
%\tableofcontents


Due to noise and decoherence from environment, the application of adiabatic protocols in quantum technologies is intensely limited. Counterdiabatic (CD) driving protocols, which is also called shortcut to adiabaticity techniques, provide a powerful alternative for controlling a quantum system when it's parameter(s) is tuned externally. These protocols allow us to change these parameters rapidly while still mimicking adiabatic dynamics. They have been shown to work well for a wide variety of systems but it's exponentially hard to find exact CD protocols for quantum many-body non-integrable systems. We study a method to approximate CD protocol which avoids exponential sensitivity to perturbations. Our finite-size scaling of counterdiabatic Hamiltonian reveals remarkable difference between quantum integrable and non-integrable systems/spin chains

%The goal, as of now, is to distinguish between integrable and non-integrable many-body quantum system by studying their approximate gauge adiabatic potential\footnote{We expect results to be valid for classical system too. But for now, we would focus on quantum systems.}

 



\bibliography{ref} 

\bibliographystyle{unsrt}
%\bibliographystyle{plain}

\end{document}
