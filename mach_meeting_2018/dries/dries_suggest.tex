\documentclass[11pt,a4paper]{article}
\usepackage[utf8]{inputenc}
\usepackage[hmargin=2.0cm,vmargin=2.5cm,bindingoffset=0.5cm]{geometry}
\usepackage{amsfonts}
\usepackage{amsmath,amsthm,amssymb}
\allowdisplaybreaks
\usepackage[english]{babel}
\author{Mohit Pandey, Dries Sels and David. K. Campbell}
\title{Approximate counter-diabatic driving protocols for quantum non-integrable systems  }
\begin{document}
\maketitle


Due to noise and decoherence from environment, the application of adiabatic protocols in quantum technologies is intensely limited. Counterdiabatic (CD) driving protocols, which also go under the name of shortcuts-to-adiabaticity, provide a powerful alternative for controlling a quantum system. These protocols allow us to change Hamiltonian parameters rapidly while still mimicking adiabatic dynamics. They have been shown to work well for a wide variety of systems but it's exponentially hard to find exact CD protocols for non-integrable quantum many-body systems. We study a method to approximate CD protocol which avoids exponential sensitivity to perturbations of the Hamiltonian. Our finite-size scaling of counterdiabatic Hamiltonian reveals remarkable difference between quantum integrable and non-integrable systems. We numerically identify different scaling regimes and show how they arise from the eigenstate thermalization hypothesis.  


\end{document}
