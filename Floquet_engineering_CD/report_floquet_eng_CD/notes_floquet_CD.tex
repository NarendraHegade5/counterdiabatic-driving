\documentclass[11pt,a4paper]{article}
\usepackage[utf8]{inputenc}
\usepackage[hmargin=2.0cm,vmargin=2.5cm,bindingoffset=0.5cm]{geometry}
\usepackage{amsfonts}
\usepackage{amsmath,amsthm,amssymb}
\allowdisplaybreaks
\usepackage{hyperref}
\usepackage{graphicx}
\usepackage{tikz}
\usepackage{mathtools}
\DeclarePairedDelimiter\ceil{\lceil}{\rceil}
\DeclarePairedDelimiter\floor{\lfloor}{\rfloor}
%\usepackage{float}
\usepackage{placeins}
\usepackage{diagbox}
\DeclareMathOperator{\Tr}{Tr}
\newtheorem{thm}{Theorem}
\usepackage{subcaption}
%\usepackage{subfigure}
\usepackage[english]{babel}
\author{Mohit}
\title{Counter-diabatic driving using Floquet engineering}
\begin{document}
\maketitle
%\tableofcontents

\section{CD driving}
 \begin{equation}
 H_0= - J \sum_j (c^{\dagger}_{i+1} c_i+ h.c) + \sum_j V_j (\lambda) c^{\dagger}_{i} c_i
 \end{equation}
For this problem, approximate gauge potential is chosen to be $A_{\lambda}^*= i \sum_j \alpha_j (c^{\dagger}_{i+1} c_i- h.c)$.

\begin{align*}
H_{CD}&= H_0 + \dot{\lambda} A_{\lambda}\\
&=-  \sum_j J_j (c^{\dagger}_{i+1} c_i+ h.c) + \sum_j U_j  c^{\dagger}_{i} c_i
\end{align*}

where 
\begin{align*}
J_j= J \sqrt{1 + (\dot{\lambda} \alpha_j/J)^2} \quad U_j = V_j( \lambda) - \sum_i^j \dfrac{J}{J^2 + (\dot{\lambda} \alpha_i/J)^2} (\ddot{\lambda} \alpha_j + \dot{\lambda}^2 \partial_{\lambda} \alpha_j)
\end{align*}
\section{Floquet driving}

 \begin{equation}
 H_0= - J \sum_j (c^{\dagger}_{i+1} c_i+ h.c) + \Omega \sum_j \dfrac{\xi}{2}  \sin (\Omega t - \Phi j + \Phi/2) c^{\dagger}_{j} c_j
 \end{equation}
Going to rotating frame defined by:
\begin{equation}
V= \exp( - i \sum_j \Delta_j c^{\dagger}_{j} c_j)
\end{equation}
where $\Delta_j= \cos (\Omega t - \Phi j + \Phi/2)$
\bibliography{ref} 

\bibliographystyle{unsrt}
%\bibliographystyle{plain}

\end{document}
