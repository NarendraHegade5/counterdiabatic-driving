\documentclass[11pt,a4paper]{article}
\usepackage[utf8]{inputenc}
\usepackage[hmargin=2.0cm,vmargin=2.5cm,bindingoffset=0.5cm]{geometry}
\usepackage{amsfonts}
\usepackage{amsmath,amsthm,amssymb}
\allowdisplaybreaks
\usepackage{hyperref}
\usepackage{graphicx}
\usepackage{tikz}
\usepackage{mathtools}
\DeclarePairedDelimiter\ceil{\lceil}{\rceil}
\DeclarePairedDelimiter\floor{\lfloor}{\rfloor}
%\usepackage{float}
\usepackage{placeins}
\usepackage{diagbox}
\DeclareMathOperator{\Tr}{Tr}
\newtheorem{thm}{Theorem}
\usepackage{subcaption}
%\usepackage{subfigure}
\usepackage[english]{babel}
\author{Mohit}
\title{Counter-diabatic driving using Floquet engineering  }
\begin{document}
\maketitle
%\tableofcontents

\section{Introduction}
$H= H_0 + H_1$
\begin{equation}
H= J\sum_j (c_{j+1}^{\dagger} c_j + h.c) + \cos(\omega t) \sum_j A_j  c_j^{\dagger} c_j
\end{equation} 

$V=\exp(-i \sin(\omega t)/ \omega \sum_j A_j  c_j^{\dagger} c_j)$ 

\begin{align*}
H_{rot}&= V^{\dagger} H V- i V^{\dagger} \dot{V}\\
&=V^{\dagger} H_0 V + \cos(\omega t) \sum_j A_j  c_j^{\dagger} c_j + i^2 \cos(\omega t) \sum_j A_j  c_j^{\dagger} c_j \\
&=V^{\dagger} H_0 V \\
&=V^{\dagger} c_{j+1}^{\dagger}V V^{\dagger} c_j V 
\end{align*}



Using $[n_j,c_j ]= -c_j$ and $[n_j,c_j^{\dagger} ]= c_j^{\dagger}$

\begin{align*}
H_{rot}&= J\sum_j   ( g^{j, j+1} c_{j+1}^{\dagger} c_j + \mbox{h.c})
\end{align*}
where $g^{j, j+1}= \exp\left(i \sin(\omega t) \dfrac{A_{j+1}- A_j}{\omega}\right)$


\begin{align*}
H_F^{(0)}&= \dfrac{1}{T}   \int_{t_0}^{T+t_0}(c_{j+1}^{\dagger} c_j \exp\left(i \sin(\omega t) \dfrac{A_{j+1}- A_j}{\omega}\right) dt + \mbox{h.c})\\
&= J_{eff} (c_{j+1}^{\dagger} c_j + \mbox{h.c})\\
\end{align*}
where $J_{eff}=J \mathcal{J}_0 \left(\dfrac{A_{j+1}- A_j}{\omega}\right)$

Few points:
Bessel function is always less than 1 while CD hopping J is always greater than 1. Furthermore, driving frequency would be high so that for all values of  $A_{j+1}- A_j$, we would have $\dfrac{A_{j+1}- A_j}{\omega} \ll 1 $ so that we are close to origin and away from zeros of Bessel's functions.
\section{Magnus expansion}
For a Hamiltonian which is periodic in time, it's unitary operator over a full driving cycle is given by:
\begin{equation}
U(T+ t_0, t_0)= \mathcal{T}_t\exp(- \dfrac{i}{\hbar} \int_{t_0}^T dt H(t))= \exp(- \dfrac{i}{\hbar}  H_F[t_0]T)
\end{equation}
$ H_F[t_0]= \sum_n H_F^{(n)}[t_0] $ 
where 
\begin{align*}
H_F^{(0)}= \dfrac{1}{T} \int_{t_0}^{T+t_0} H(t) dt \\
H_F^{(1)}= \dfrac{1}{2! T i \hbar} \int_{t_0}^{T+t_0}  dt_1\int_{t_0}^{t_1} dt_2  [H(t_1), H(t_2)] 
\end{align*}
\section{Bessel's function of first kind}
Integral representation of Bessel's function of first kind $\mathcal{J}_n (x)$ is given by:
\begin{equation}
\mathcal{J}_n (x)=  \frac{1}{2 \pi} \int_{-\pi}^\pi d\tau e^{i(n \tau - x \sin \tau)}= \frac{1}{T} \int_{-T}^T d\tau e^{i(n   \omega \tau - x \sin \omega \tau)}
\end{equation}
For $x \ll1$, $\mathcal{J}_0 (x)= 1- \frac{x^2}{2}$
\bibliography{ref} 

\bibliographystyle{unsrt}
%\bibliographystyle{plain}

\end{document}
